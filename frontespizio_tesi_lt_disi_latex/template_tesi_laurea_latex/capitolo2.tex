\chapter{Data collection}
\label{cha:data}
The dataset used for the project is the \textbf{echen102/COVID-19-TweetIDs} GitHub repository\cite{chen2020tracking}. The repository contains an ongoing collection of tweets IDs, starting on the 28th of January 2020, from specified accounts and also real-time tweets that mention specific keywords.

\begin{table}[h]
    \centering
    \begin{tabularx}{\textwidth}{lXr}
        Number of files & & \textbf{10 402}
        \\ \lightrule
        Number of identified languages & & \textbf{65}
        \\ \lightrule
        Number of tweets & & \textbf{1 055 843 481}
        \\ \lightrule
        Number of unique tweets (no retweets) & & \textbf{323 504 667}
        \\ \lightrule
        Dataset compressed size & & \textbf{865 GB}
        \\ \lightrule
        Dataset estimated uncompressed size & & \textbf{6.252 TB}
    \end{tabularx}
    \caption{Dataset general statistics}
    \label{tab:dataset-stats}
\end{table}

\begin{table}[h]
    \centering
    \begin{tabularx}{\textwidth}{XXR{3cm}R{2.5cm}R{2.5cm}R{2.5cm}}
    		\textbf{language} & \textbf{ISO} & \textbf{unique tweets} & \textbf{retweets} & \textbf{total} & \textbf{percentage} \\
    		\midrule
        English & en & 195 645 826 & 473 950 322 & 669 596 148 & 63.41\% \\
		\lightrule
		Spanish & es & 35 533 886 & 111 464 189 & 146 998 075 & 13.92\% \\
		\lightrule
		Portuguese & pt & 15 459 760 & 29 912 427 & 45 372 187 & 4.30\% \\
		\lightrule
		French & fr & 9 547 251 & 23 635 273 & 33 182 524 & 3.14\% \\
		\lightrule
		Undefined & und & 20 560 392 & 8 590 707 & 29 151 099 & 2.76\% \\
		\lightrule
		Indonesian & in & 9 029 012 & 16 479 537 & 25 508 549 & 2.41\% \\
		\lightrule
		German & de & 8 091 516 & 11 447 554 & 19 539 070 & 1.85\% \\
		\lightrule
		Japanese & ja & 3 228 542 & 10 220 609 & 13 449 151 & 1.27\% \\
		\lightrule
		Italian & it & 5 256 748 & 7 173 234 & 12 429 982 & 1.18\% \\
		\lightrule
		Turkish & tr & 3 347 597 & 6 698 252 & 10 045 849 & 0.95\%
    \end{tabularx}
    \caption{Top 10 languages with the most tweets}
    \label{tab:dataset-stats}
\end{table}

\subsection{Tweets}

To comply with Twitter's Term of Service, tweets cannot be released publicly: the repository is in fact a collection of tweets IDs. The original tweets can be retrieved, or hydrated, using the Python library Twarc with a Twitter Developer Account. Given an id, Twarc simply uses the token of the associated developer account to contact the API, and returns the corresponding tweet as a json object.

The original structure of the tweets was changed, in order to consider only the relevant fields:

\begin{lstlisting}[language=json]
{
  "id": 1307025659294674945,
  "full_text": "Here's an article that highlights the updates...",
  "lang": "en",
  "created_at": "Fri Sep 18 18:36:15 +0000 2020",
  "retweet_count": 11,
  "favorite_count": 70,
  "user": {
    "id": 2244994945,
    "id_str": "2244994945",
    "screen_name": "TwitterDev",
    "name": "Twitter Dev",
    "description": "The voice of the #TwitterDev team and your official...",
    "location": "127.0.0.1",
    "followers_count": 513958,
    "statuses_count": 3635,
    "default_profile_image": false,
    "profile_image_url_https": "https:\/\/pbs.twimg.com\/profile_images\/1283786620521652229\/lEODkLTh_normal.jpg"
  }
}
\end{lstlisting}

\subsection{Analyzed period and languages}

%

We have decided to consider the period from January 2020 to March 2021


\section{Cras in aliquam quam, et}
\label{sec:456}
Lorem ipsum dolor sit amet, consectetur adipiscing elit. Donec sed nunc orci. Aliquam nec nisl vitae sapien pulvinar dictum quis non urna. Suspendisse at dui a erat aliquam vestibulum. Quisque ultrices pellentesque pellentesque. Pellentesque egestas quam sed blandit tempus. Sed congue nec risus posuere euismod. Maecenas ut lacus id mauris sagittis egestas a eu dui. Class aptent taciti sociosqu ad litora torquent per conubia nostra, per inceptos himenaeos. Pellentesque at ultrices tellus. Ut eu purus eget sem iaculis ultricies sed non lorem. Curabitur gravida dui eget ex vestibulum venenatis. Phasellus gravida tellus velit, non eleifend justo lobortis eget.


\subsection{Sed pulvinar placerat enim, a}
\label{sec:00456}
Lorem ipsum dolor sit amet, consectetur adipiscing elit. Donec sed nunc orci. Aliquam nec nisl vitae sapien pulvinar dictum quis non urna. Suspendisse at dui a erat aliquam vestibulum. Quisque ultrices pellentesque pellentesque. Pellentesque egestas quam sed blandit tempus. Sed congue nec risus posuere euismod. Maecenas ut lacus id mauris sagittis egestas a eu dui. Class aptent taciti sociosqu ad litora torquent per conubia nostra, per inceptos himenaeos. Pellentesque at ultrices tellus. Ut eu purus eget sem iaculis ultricies sed non lorem. Curabitur gravida dui eget ex vestibulum venenatis. Phasellus gravida tellus velit, non eleifend justo lobortis eget.


\section{Vivamus hendrerit imperdiet ex. Vivamus}
\label{sec:123}
Lorem ipsum dolor sit amet, consectetur adipiscing elit. Donec sed nunc orci. Aliquam nec nisl vitae sapien pulvinar dictum quis non urna. Suspendisse at dui a erat aliquam vestibulum. Quisque ultrices pellentesque pellentesque. Pellentesque egestas quam sed blandit tempus. Sed congue nec risus posuere euismod. Maecenas ut lacus id mauris sagittis egestas a eu dui. Class aptent taciti sociosqu ad litora torquent per conubia nostra, per inceptos himenaeos. Pellentesque at ultrices tellus. Ut eu purus eget sem iaculis ultricies sed non lorem. Curabitur gravida dui eget ex vestibulum venenatis. Phasellus gravida tellus velit, non eleifend justo lobortis eget.


