

\chapter{Conclusions}
\label{cha:conclusions}

\paragraph{Results}

The analysis of the data revealed some first interesting results:

\begin{itemize}
	\item from the comparison of the results obtained from LIWC and EmoLex we noticed that on several occasions the two lines overlap. Moreover, even if there is a gap, they generally tend to increase or decrease simultaneously, meaning that EmoLex could be used for those languages where LIWC is not available.
	\item there are cases when the course of the emotions seems to be the same. In particular, when one emotion increases, so do the others. This probably happens because on some days users are simply more emotional and tend to use more words. Because of that, the probability of conveying more emotions increases. This particular behavior mostly regarded EmoLex, where even opposite emotions could be associated with a single word.
	\item if we consider the English tweets, women tend to write more tweets conveying sadness. This seems to be also the case for anxiety and positive emotions. However, the gap becomes less and less significant. Instead, men seem to express slightly more anger over the considered period, but the two lines overlap on more than one occasion.
	\item again from the analysis of the English tweets, we noticed that users below forty years old seem to be slightly more aggressive. Instead, those at least forty years old express more sadness, anxiety, and positive emotions with a noticeable gap. The fact that users below forty years old express fewer emotions could be because they prefer to write shorter tweets, use many emoticons or slang, or are simply more inexpressive.
	\item finally, we were able to analyze the course of the emotions expressed by users belonging to each Italian region. Moreover, we were able to notice how, after the discovery of the first case, people's anxiety dropped. This probably happened because news and various decrees helped to calm people down.
\end{itemize}

\paragraph{Contributions}

I have contributed to the project by
    
\begin{itemize}
    	\item retrieving over 6TB of data from Twitter
    	\item analyzing users' emotions using state-of-the-art libraries (EmoLex, LIWC)
    	\item studying which emotions were expressed by a large group of users in 4 languages (ca, en, es, it)
    	\item studying how the emotions expressed by users can vary based on their demographic characteristic, including inferred age and gender
    	\item studying how the emotions expressed by users can vary based on their geographical provenance (at national and regional level)
\end{itemize}

Furthermore, I am particularly proud of my personal contribution to improve m3inference. In practice, I opened a pull request on GitHub to solve some issues while downloading images from Twitter.

In any case, I hope that the effort that I put into the project could be a good starting point for further studies.

\paragraph{Further developments}

I was only able to scratch the surface of this research field because the amount of data to analyze was really impressive. Aside from the results obtained, it must be underlined that there is always some space left for improvement that could bring even more value to this research.

First of all, the data were analyzed using both EmoLex and LIWC. However, none of them considers the context of the words. In fact, in EmoLex there are words such as “hospital” that tend to convey both negative and positive emotions, which introduces a bias in our results. On the other hand, even with LIWC some mistakes are possible: for example, the most used word in the Italian tweets is “solo”. While this word could be used to convey a sense of loneliness as an adjective, it could be also be used as an adverb or a noun. In general, NLP algorithms (e.g. opinion mining algorithms) could be used to consider the context of a specific word in a phrase to reduce the valid subset of emotions and minimize the false positives.

Secondly, instead of taking into account all the possible emotions expressed in a single tweet, it may be more useful to consider the user's phrase as a whole. In practice, a prediction score could be assigned to the tweet for each possible emotion. Then, the tweet should be evaluated by considering the emotion with the highest confidence. Once again, this process could be achieved using NLP algorithms. However, this is no simple task at all, because human language is extremely complex.

Finally, the purpose of the project is to understand which restrictive measures were more welcomed than others, in order to handle in a better way the situation in the case of some other unfortunate event. However, to associate a certain emotional reaction on a given week to a specific event, it is necessary to have a reliable events database. For this particular reason, we are currently looking for the best methodology, through the analysis of related work, to obtain valid data. 