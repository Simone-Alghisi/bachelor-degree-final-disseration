\chapter{Introduction}
\label{cha:intro}

The COVID-19 pandemic is having a huge impact on our lives, that goes beyond the direct effects of the virus. Besides the fear of infection, lockdown measures adopted by many countries are limiting the possibility to move, work, have contact with others, and are creating a situation of economic crisis and generalized uncertainty about the future. The psychological effects of this unprecedented situation need to be studied.


\section{Context and motivations}
\label{sec:context}

During this year, everyone daily life changed significantly and we had to adapt to restrictive measures in order to stop the disease: whether we liked it or not. This research proposed by Eurecat really caught my eye: the possibility to study how people perceived all of this situation, and better understand which measures were more welcomed than others, was really fascinating and, above all, may be useful in the case of some other unfortunate event.

\section{Project description}
\label{sec:project}

The project consisted in an analysis of emotions as emerging from Twitter messages during the pandemic.

Lexicon-based sentiment analysis tools have been employed to characterize emotions associated with content on a large scale. Moreover, users have been divided into into two different groups w.r.t. their gender, to study the different emotional response of males and females, and also based on their location. 

This could allow us to contrast the emotional reaction with the evolution of contagions and deaths, and with the different lockdown and de-escalation stages, in different areas.

